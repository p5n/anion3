
\chapter{Basic configuration}
\label{chap:config}

This chapter should help your configure Ion to your liking. As  the your
probably already know, Ion uses Lua as a configuration and extension 
language. If you're new to it, you might first want to read some Lua 
documentation as already suggested and pointed to in the Introduction
before continuing with this chapter.

Section \ref{sec:conffiles}�is an overview of the multiple configuration
files Ion uses and as a perhaps more understandable introduction to the
general layout of the configuration files, a walk-through of the main 
configuration file \file{cfg\_ion.lua} is provided in section 
\ref{sec:walkthrough}.
How keys and mouse action are bound to functions is described in detail
in \ref{sec:bindings} and in section \ref{sec:winprops} winprops are
explained. Finally, the statusbar is explained in \ref{sec:statusbar}.
For a reference on exported functions, see section \ref{sec:exports}.

\section{The configuration files}
\label{sec:conffiles}

Ion3, to which document applies, stores its stock configuration files in
\file{/usr/local/etc/ion3/} unless you, the OS package maintainer or 
whoever  installed the package on the system has modified the variables
\code{PREFIX}\index{PREFIX@\code{PREFIX}} or
\code{ETCDIR}\index{ETCDIR@\code{ETCDIR}} in
\file{system.mk}\index{system.mk@\file{system.mk}} before compiling Ion.
In the first case you probably know where to find the files and in 
the other case the system administrator or the OS package maintainer
should  have provided documentation to point to the correct location. 
If these instructions are no help in locating the correct directory, 
the command \code{locate cfg_ion.lua} might help provided \code{updatedb} 
has been run recently. 

User configuration files go in \file{\~{}/.ion3/}. 
Ion always searches the user configuration file directory before the stock
configuration file directory for files. Therefore, if you want to change
some setting, it is advised against that you modify the stock configuration
files in-place as subsequent installs of Ion will restore the stock
configuration files. Instead you should always make a copy of the stock
file in \file{\~{}/.ion3/} and modify this file. For sake of maintainability
of your customised configuration, it is recommended against copying all of
the files there. Only copy those files you actually need to modify. Most 
simple customisations, such as changes in a few bindings, are best done 
entirely within \file{cfg\_ion.lua}.

All the configuration files are named \file{cfg\_*.lua} with the ``\file{*}''
part varying. The configuration file for each module \file{mod\_modname} is
\file{cfg\_modname.lua}, with \file{modname} varying by the module in
question. Configuration files can also be compiled into \file{.lc} files,
and these are attempted by the configuration file search routines before
\file{.lua} files.

The following table summarises these and other configuration
files:

\begin{tabularx}{\linewidth}{
      p{\widthof{cfg-bindings.lua}}%
      X}
    \hline
    File & Description \\
    \hline
    \file{cfg\_ion.lua} & 
    The main configuration file \\
    %
    \file{cfg\_ioncore.lua} & 
    Configuration file for Ion's core library.
    Most of the bindings and menus are configured here. Bindings that are
    specific to some module are configured in the module's configuration
    file. For details, see section \ref{sec:bindings}. \\
    %
    \file{cfg\_kludges.lua} & 
    Settings to get some applications behave more nicely have been 
    collected here. See section \ref{sec:winprops}. \\
    %
    \file{cfg\_layouts.lua} & 
    Some workspace layouts are defined here. \\
    %
    \file{cfg\_tiling.lua} 
    \file{cfg\_query.lua} 
    \file{cfg\_menu.lua} 
    \file{cfg\_dock.lua} 
    \file{cfg\_statusbar.lua} 
    \dots & Configuration files for different modules. \\
\end{tabularx}

Additionally, there's the file \file{look.lua} that configures the 
drawing engine, but it is covered in chapter \ref{chap:gr}.

\section{A walk through \file{cfg\_ion.lua}}
\label{sec:walkthrough}

As already mentioned \file{cfg\_ion.lua} is Ion's main configuration
file. Some basic 'feel' settings are usually configured there and
the necessary modules and other configuration files configuring some 
more specific aspects of Ion are loaded there. In this section we
take a walk through the stock \file{cfg\_ion.lua}.
Notice that most of the settings are commented-out (\verb!--! is a 
line comment in Lua) in the actual file, as they're the defaults
nevertheless.

The first thing done in the file, is to set
\begin{verbatim}
META="Mod1+"
ALTMETA=""
\end{verbatim}
These settings cause most of Ion's key bindings to use \key{Mod1} as the
modifier key. If \code{ALTMETA} is set, it is used as modifier for the
keys that don't normally use a modifier. Note that these two are Lua 
variables used in the configuration files only, and not Ion settings. 
For details on modifiers and key binding setup in general, see section
\ref{sec:bindings}.

Next we do some basic feel configuration:

\begin{verbatim}
ioncore.set{
    dblclick_delay=250,
    kbresize_delay=1500,
}
\end{verbatim}

These two will set the delay between button presses in a double click, and
the timeout to quit resize mode in milliseconds.

\begin{verbatim}
ioncore.set{
    opaque_resize=true,
    warp=true
}
\end{verbatim}

The first of these two settings enables opaque resize mode: in move/resize
move frames and other objects mirror you actions immediately. If opaque
resize is disabled, a XOR rubber band is shown during the mode instead.
This will, unfortunately, cause Ion to also grab the X server and has some
side effects. 

There are some other options as well; see the documentation
for \fnref{ioncore.set} for details.

As a next step, in the actual \file{cfg\_ion.lua} file, we load
\file{cfg\_defaults.lua}. However, it is merely a convenience file for
doing exactly what we will going through below, and what is commented
out in the actual file. If you do not want to load what 
\file{cfg\_defaults.lua} loads, just comment out the corresponding 
line, and uncomment the lines for the files that you want:

\begin{verbatim}
--dopath("cfg_defaults")
dopath("cfg_ioncore")
dopath("cfg_kludges")
dopath("cfg_layouts")
\end{verbatim}

Most bindings and menus are defined in \file{cfg\_ioncore.lua}.
Details on making such definitions follow in sections \ref{sec:bindings} 
and \ref{sec:menus}, respectively. 
some kludges or ``winprops'' to make some applications behave better
under Ion are collected in \file{cfg\_kludges.lua}; see section
\ref{sec:winprops} for details. In addition to these, this file
lists quite a few statements of the form
\begin{verbatim}
ioncore.defshortening("[^:]+: (.*)(<[0-9]+>)", "$1$2$|$1$<...$2")
\end{verbatim}
These are used to configure how Ion attempts to shorten window titles
when they do not fit in a Tab. The first argument is a POSIX regular
expression that is used to match against the title and the next is
a rule to construct a new title of a match occurs. This particular
rule is used to shorten e.g. 'Foo: barbaz<3>' to 'barba{\ldots}<3>'; for
details see the function reference entry for \fnref{ioncore.defshortening}.
Finally, \file{cfg\_layouts.lua} defines some workspace layouts, available
through the \key{F9} workspace creation query.

To actually be able to do something besides display windows in full screen
mode, we must next load some modules:

\begin{verbatim}
dopath("mod_query")
dopath("mod_menu")
dopath("mod_tiling")
dopath("mod_statusbar")
--dopath("mod_dock")
dopath("mod_sp")
\end{verbatim}


\section{Keys and rodents}
\label{sec:bindings}

In the stock configuration file setup, most key and mouse bindings are set
from the file \file{cfg\_ioncore.lua} while module-specific bindings
are set from the modules' main configuration files (\file{cfg\_modname.lua}).
This, however, does not have to be so as long as the module has been
loaded prior to defining any module-specific bindings.

Bindings are defined by calling the function 
\fnrefx{ioncore}{defbindings} with the ``context'' of the
bindings and the a table of new bindings to make. The context is simply
string indicating one of the classes of regions (or modes such as
\type{WMoveresMode}) introduced in section \ref{sec:objects}, and fully
listed in appendix \ref{app:fullhierarchy}, although not all define
a binding map. For example, the following skeleton would be used to 
define new bindings for all frames:

\begin{verbatim}
defbindings("WFrame", {
    -- List of bindings to make goes here.
})
\end{verbatim}

There has been some confusion among users about the need to define the
``context'' for each binding, so let me try to explain this design
decision here. The thing is that if there was a just a simple 'bind this 
key to this action' method without knowledge of the context, some 
limitations would have to be made on the available actions and writing 
custom handlers would be more complicated. In addition one may want to 
bind the same function to different key for different types of objects.
Indeed, the workspace and frame tab switching functions are the same both
classes being based on \type{WMPlex}, and in the stock configuration the 
switch to $n$:th workspaces is bound to \key{Mod1+n} while the switch to 
$n$:th tab is bound to the sequence \key{Mod1+k n}.

Currently known contexts include: 
\codestr{WScreen},
\codestr{WMPlex},
\codestr{WMPlex.toplevel},
\codestr{WFrame},
\codestr{WFrame.toplevel},
\codestr{WFrame.floating},
\codestr{WFrame.tiled},
\codestr{WFrame.transient},
\codestr{WMoveresMode},
\codestr{WGroup},
\codestr{WGroupCW},
\codestr{WGroupWS},
\codestr{WClientWin},
\codestr{WTiling}, and
\codestr{WStatusBar}.
Most of these should be self-explanatory, corresponding to objects
of class with the same name. The ones with \codestr{.toplevel} suffix
refer to screens and ``toplevel''  frames, i.e. frames that are
not used for transient windows. Likewise \codestr{.transient} refers
to frames in transient mode, and \codestr{.tiled} and \codestr{.floating}
to frames in, respectively, tiled and floating modes. 

                                                        

The following subsections describe how to construct elements of the
binding table. Note that \fnrefx{ioncore}{defbindings} adds
the the newly defined bindings to the previous bindings of the context,
overriding duplicates. To unbind an event, set the handler parameter
to \code{nil} for each of the functions to be described in the following
subsections.

Also note that when multiple objects want to handle a binding, the 
innermost (when the root window is considered the outermost) active object
in the parent--child hierarchy (see Figure \ref{fig:parentship}) of objects 
gets to handle the action.


\subsection{Binding handlers and special variables}

Unlike in Ion2, in Ion3 binding handlers are not normally passed as
``anonymous functions'', although this is still possible. The preferred
method now is to pass the code of the handler as a string. Two following
special variables are available in this code.

\begin{tabularx}{\linewidth}{lX}
    \tabhead{Variable & Description}
    \code{_} (underscore) &
      Reference to the object on which the 
      binding was triggered. The object is of the same class as the the
      context of the \fnrefx{ioncore}{defbindings} call
      defining the binding. \\
    \code{_sub} &
      Usually, the currently active \emph{managed object} of the 
      object referred to by \code{_}, but sometimes (e.g. mouse actions
      on tabs of frames) something else relevant to the action triggering
      the binding. \\
    \code{_chld} &
      Object corresponding to the currently active child window of the
       object referred to by \code{_}. This should seldom be needed.
\end{tabularx}

For example, supposing \code{_} (underscore) is a \type{WFrame}, the 
following handler should move the active window to the right, if 
possible:

\begin{verbatim}
"_:inc_index(_sub)"
\end{verbatim}

\subsection{Guards}

To suppress error messages, each binding handler may also be accompanied
by a ``guard'' expression that blocks the handler from being called when
the guard condition is not met. Currently the following guard expressions
are supported (for both \code{_sub} and \code{_chld}):

\begin{tabularx}{\linewidth}{lX}
    \tabhead{Guard & Description}
    \codestr{\_sub:non-nil} & The \code{_sub} parameter must be set. \\
    \codestr{\_sub:SomeClass} & The \code{_sub} parameter must be member
      of class \type{SomeClass}. \\
\end{tabularx}


\subsection{Defining the bindings}
\label{sec:binddef}

The descriptions of the individual bindings in the binding table argument
to \fnrefx{ioncore}{defbindings} should be constructed with the following
functions.

Key presses:
\begin{itemize}
    \item \fnrefx{ioncore}{kpress}, and
          \fnrefx{ioncore}{kpress_wait}\code{(keyspec, handler [, guard])}.
    \item \fnrefx{ioncore}{submap}\code{(keyspec, \{ ... more key bindings ... \})}.
    \item \fnrefx{ioncore}{submap_enter}, and
          \fnrefx{ioncore}{submap_wait}\code{(handler [, guard])}.
\end{itemize}
Mouse actions:
\begin{itemize}
    \item \fnrefx{ioncore}{mclick},
          \fnrefx{ioncore}{mdblclick},
          \fnrefx{ioncore}{mpress}, and
          \fnrefx{ioncore}{mdrag}\code{(buttonspec, handler [, guard])}.
\end{itemize}

The actions that most of these functions correspond to should be clear
and as explained in the reference, \fnrefx{ioncore}{kpress_wait} is simply
\fnrefx{ioncore}{kpress} with a flag set instructing Ioncore wait for
all modifiers to be released before processing any further actions.
This is to stop one from accidentally calling e.g.
\fnref{WRegion.rqclose} multiple times in a row. The 
\fnrefx{ioncore}{submap} function is used to define submaps or
``prefix maps''. The second argument to this function is table listing
the key press actions (\fnrefx{ioncore}{kpress}) in the submap. 
The \fnrefx{ioncore}{submap_enter} handler is called when the submap
is entered, in which this handler is defined. Likewise, the
\fnrefx{ioncore}{submap_wait} handler is  called when all modifiers
have been released while waiting for further key presses in the submap.

The parameters \var{keyspec} and \var{buttonspec} are explained below
in detail. The parameter \var{handler} is the handler for the binding,
and the optional parameter \var{guard} its guard. These should normally
be strings as explained above. 

\subsection{Examples}

For example, to just bind the key \key{Mod1+1} to switch to the first
workspace and \key{Mod1+Right} to the next workspace, you would make the
following call
\begin{verbatim}
defbindings("WScreen", {
    kpress("Mod1+Right", "_:switch_next()"),
    kpress("Mod1+1", "_:switch_nth(1)"),
})
\end{verbatim}

Note that \code{_:switch_nth(1)} is the same as calling
\fnref{WMPlex.switch_next}\code{(_, 1)} as \type{WScreen} inherits
\type{WMPlex} and this is where the function is actually defined.

Similarly to the above example, to bind the key sequence \key{Mod1+k n} 
switch to the next managed object within a frame, and \key{Mod1+k 1} to the
first, you would issue the following call:
\begin{verbatim}
defbindings("WFrame", {
    submap("Mod1+K", {
        kpress("Right", "_:switch_next()"),
        kpress("1", "_:switch_nth(1)"),
   }),
})
\end{verbatim}


\subsection{Key specifications}

As seen above, the functions that create key binding specifications require
a \var{keyspec} argument. This argument should be a string containing the
name of a key as listed in the X header file \file{keysymdef.h}%
\footnote{This file can usually be found in the directory
\file{/usr/X11R6/include/X11/}.} without the \code{XK_} prefix.
\index{keysymdef.h@\file{keysymdef.h}}
Most of the key names are quite intuitive while some are not. For example,
the \key{Enter} key on the main part of the keyboard has the less common
name \key{Return} while the one the numpad is called \key{KP\_Enter}.

The \var{keyspec} string may optionally have multiple ``modifier'' names
followed by a plus sign (\code{+}) as a prefix. X defines the following
modifiers:

\key{Shift}, \key{Control}, \key{Mod1} to \key{Mod5},
\key{AnyModifier} and \key{Lock}.
\index{Shift@\key{Shift}}
\index{Control@\key{Control}}
\index{ModN@\key{ModN}}
\index{AnyModifier@\key{AnyModifier}}
\index{Lock@\key{Lock}}

X allows binding all of these modifiers to almost any key and while this
list of modifiers does not explicitly list keys such as 
\key{Alt}\index{Alt@\key{Alt}} that are common on modern keyboards, such
keys are bound to one of the \key{ModN}. On systems running XFree86
\key{Alt} is usually \key{Mod1}. On Suns \key{Mod1} is the diamond key
and \key{Alt} something else. One of the ``flying window'' keys on so
called Windows-keyboards is probably mapped to \key{Mod3} if you have
such a key. Use the program \file{xmodmap}\index{xmodmap@\file{xmodmap}}
to find out what exactly is bound where. 

Ion defaults to \key{AnyModifier} in submaps. This can sometimes lead to
unwanted effects when the same key is used with and without explicitly
specified modifiers in nested regions. For this reason, Ion recognises
\key{NoModifier} as a special modifier that can be used to reset this
default.

Ion ignores the \key{Lock} modifier and any \key{ModN} ($N=1{\ldots} 5$)
bound to \key{NumLock}\index{NumLock@\key{NumLock}} or
\key{ScrollLock}\index{ScrollLock@\key{ScrollLock}}
by default because such\footnote{Completely useless keys that should be
gotten rid of in the author's opinion.} locking keys may otherwise
cause confusion.


\subsection{Button specifications}

Button specifications are similar to key definitions but now
instead of specifying modifiers and a key, you specify modifiers
and one of the button names \key{Button1} to
\key{Button5}\index{Button-n@\key{Button-n}}. Additionally the
specification may end with an optional area name following an @-sign.
Only frames currently support areas, and the supported values in this
case are
\codestr{border}, \codestr{tab}, \codestr{empty\_tab}, \codestr{client} 
and \code{nil} (for the whole frame).

For example, the following code binds dragging a tab with the first 
button pressed to initiate tab drag\&drop handling:

\begin{verbatim}
defbindings("WFrame", {
    mdrag("Button1@tab", "_:p_tabdrag()"),
})
\end{verbatim}


\subsection{A further note on the default binding configuration}

The default binding configuration contains references to the variables
\code{META} and \code{ALTMETA} instead of directly using the default
values of \codestr{Mod1+} and \codestr{} (nothing). As explained in
section \ref{sec:walkthrough}, the definitions of these variables
appear in \file{cfg\_ion.lua}. This way you can easily change the the
modifiers used by all bindings in the default configuration without 
changing the whole binding configuration. Quite a few people prefer 
to use the Windows keys as modifiers because many applications already
use \key{Alt}. Nevertheless, \key{Mod1} is the default as a key bound 
to it is available virtually everywhere.



\section{Menus}
\label{sec:menus}

\subsection{Defining menus}

\index{menus}
\index{defmenu@\code{defmenu}}
\index{menuentry@\code{menuentry}}
\index{submenu@\code{submenu}}
In the stock configuration file setup, menus are defined in the file
\file{cfg\_menus.lua} as previously mentioned. The \file{mod\_menu} module
must be loaded for one to be able to define menus, and this is done with
the function \fnrefx{mod_menu}{defmenu} provided by it.

Here's an example of the definition of a rather simple menu with a submenu:

\begin{verbatim}
defmenu("exitmenu", {
    menuentry("Restart", "ioncore.restart()"),
    menuentry("Exit", "ioncore.shutdown()"),
})

defmenu("mainmenu", {
    menuentry("Lock screen", "ioncore.exec('xlock')"),
    menuentry("Help", "mod_query.query_man(_)"),
    submenu("Exit", "exitmenu"),
})
\end{verbatim}


The \fnrefx{mod_menu}{menuentry} function is used to create an entry in the 
menu with a title and an entry handler to be called when the menu entry
is activated. The parameters to the handler are similar to those of binding
handlers, and usually the same as those of the binding that opened the menu.

The \fnrefx{mod_menu}{submenu} function is used to insert a submenu at that 
point in the menu. (One could as well just pass a table with the menu
entries, but it is not encouraged.)

\subsection{Special menus}

The menu module predefines the following special menus. These can be used
just like the menus defined as above.

\begin{tabularx}{\linewidth}{lX}
    \tabhead{Menu name & Description}
    \codestr{windowlist} & 
    List of all client windows. Activating an entry jumps to that window. \\
    \codestr{workspacelist} & 
    List of all workspaces. Activating an entry jumps to that workspaces. \\
    \codestr{focuslist} & 
    List of client windows with recent activity in them, followed by 
    previously focused client windows. \\
    \codestr{focuslist\_} & 
    List of previously focused client windows. \\
    \codestr{stylemenu} &
    List of available \file{look\_*.lua} style files. Activating an entry
    loads that style and ask to save the selection. \\
    \codestr{ctxmenu} &
    Context menu for given object. \\
\end{tabularx}


\subsection{Defining context menus}

The ``ctxmenu'' is a special menu that is assembled from a defined context
menu for the object for which the menu was opened for, but also includes
the context menus for the manager objects as submenus.

Context menus for a given region class are defined with the
\fnrefx{mod_menu}{defctxmenu} function. This is other ways similar to
\fnrefx{mod_menu}{defmenu}, but the first argument instead being the name
of the menu, the name of the region class to define context menu for.
For example, here's part of the stock \type{WFrame} context menu 
definition:

\begin{verbatim}
defctxmenu("WFrame", {
    menuentry("Close", "WRegion.rqclose_propagate(_, _sub)"),
    menuentry("Kill",  "WClientWin.kill(_sub)", "_sub:WClientWin"),
})
\end{verbatim}

Some of the same ``modes'' as were available for some bindings
may also be used: \codestr{WFrame.tiled}, \codestr{WFrame.floating},
and \codestr{WFrame.transient}.


\subsection{Displaying menus}
\label{sec:menudisp}

The following functions may be used to display menus from binding
handlers (and elsewhere):

\begin{tabularx}{\linewidth}{lX}
    \tabhead{Function & Description}
    \fnref{mod_menu.menu} &
      Keyboard (or mouse) operated menus that open in the bottom-left corner
      of a screen or frame. \\
    \fnref{mod_menu.pmenu} &
      Mouse-operated drop-down menus. This function can only be called from a
      mouse press or drag handler. \\
    \fnref{mod_menu.grabmenu} &
      A special version of \fnref{mod_menu.menu} that grabs the keyboard
      and is scrolled with a given key until all modifiers have been released,
      after which the selected entry is activated. \\
\end{tabularx}

Each of these functions takes three arguments, which when
called from a binding handler, should be the parameters to the handler, and
the name of the menu. For example, the following snippet of of code binds
the both ways to open a context menu for a frame:

\begin{verbatim}
defbindings("WFrame", {
    kpress(MOD1.."M", "mod_menu.menu(_, _sub, 'ctxmenu')"),
    mpress("Button3", "mod_menu.pmenu(_, _sub, 'ctxmenu')"),
})
\end{verbatim}


\section{Winprops}
\label{sec:winprops}

The so-called ``winprops''\index{Winprops} can be used to change how
specific windows are handled and to set up some kludges to deal with
badly behaving applications. They are defined by calling the function
\code{defwinprop} with a table containing the properties to set and the
necessary information to identify a window. The currently supported
winprops are listed below, and the subsequent subsections explain the
usual method of identifying windows, and how to obtain this information.

%\begin{table}
%\begin{htmlonly}
%\docode % latex2html kludge
%\end{htmlonly}
%\caption{Supported winprops}
%\label{tab:winprops}

\newenvironment{winprop}[2]{
  \begin{function}%
      % Sigh. (La)TeX is a mess.
      %\index{%
      %  \ifx\\#1\\%
      %  #2\else#1\fi%
      %  @\expandafter\var{#2}}
      \item[Winprop:] \var{#1} (#2)
      \item[Description:]
}
{
  \end{function}
}


\begin{winprop}{acrobatic}{boolean}
    \index{acrobatic@\var{acrobatic}}
    Set this to \code{true} for Acrobat Reader. It has an annoying
    habit of trying to manage its dialogs instead of setting them as
    transients and letting the window manager do its job, causing
    Ion and acrobat go a window-switching loop when a dialog is
    opened. 
\end{winprop}


\begin{winprop}{float}{boolean}
    \index{float@\var{float}}
    Set this to open the window in a floating frame, when
    in a group.
\end{winprop}


\begin{winprop}{fullscreen}{boolean}
    \index{fullscreen@\var{fullscreen}}
    Should the window be initially in full screen mode?
\end{winprop}


\begin{winprop}{ignore_cfgrq}{boolean}
    \index{ignore-cfgrq@\var{ignore_cfgrq}}
    Should configure requests on the window be ignored?
    Only has effect on floating windows.
\end{winprop}


\begin{winprop}{ignore_net_active_window}{boolean}
    \index{ignore-net-active-window@\var{ignore_net_active_window}}
    Ignore extended WM hints \code{_NET_ACTIVE_WINDOW} request. 
\end{winprop}


\begin{winprop}{jumpto}{boolean}
    \index{jumpto@\var{jumpto}}
    Should a newly created client window always be made
    active, even if the allocated frame isn't. 
\end{winprop}


\begin{winprop}{new_group}{string}
    \index{new-group@\var{new_group}}
    If the region specified by \code{target} winprop does not exist
    (or that winprop is not set), create a new workspace using the 
    previously stored layout (see \fnref{ioncore.deflayout}) named by
    this property. After creating the workspace, \code{target} is 
    attempted to be found again. (If that still fails, the newly 
    created workspace is still asked to manage the client window.)
\end{winprop}


\begin{winprop}{oneshot}{boolean}
    \index{oneshot@\var{oneshot}}
    Discard this winprop after first use. 
\end{winprop}


\begin{winprop}{orientation}{string}
    \index{orientation@\var{orientation}}
    The orientation of the window: one of \codestr{vertical} or
    \codestr{horizontal}. This is only useful when using the
    window as a status display.
\end{winprop}


\begin{winprop}{statusbar}{string}
    \index{statusbar@\var{statusbar}}
    Put the window in the statusbar, in the named tray component,
    (The default tray component is called simply \codestr{systray}, 
    and others you give names to in your custom template, always 
    prefixed by \codestr{systray\_}.
\end{winprop}


\begin{winprop}{switchto}{boolean}
    \index{switchto@\var{switchto}}
    Should a newly mapped client window be switched to within
    its frame.
\end{winprop}
        

\begin{winprop}{target}{string}
    \index{target@\var{target}}
    The name of an object (workspace, frame) that should manage 
    windows of this type. See also \code{new_group}.
\end{winprop}

        
\begin{winprop}{transient_mode}{string}
    \index{transient-mode@\var{transient_mode}}
    \codestr{normal}: No change in behaviour. \codestr{current}:
    The window should be thought of as a transient for the current
    active client window (if any) even if it is not marked as a
    transient by the application. \codestr{off}: The window should 
    be handled as a normal window even if it is marked as a
    transient by the application. 
\end{winprop}


\begin{winprop}{transparent}{boolean}
    \index{transparent@\var{transparent}}
    Should frames be made transparent when this window is selected? \\
\end{winprop}


\subsection{Sizehint winprops}

Additionally, the winprops 
\code{max_size}\index{max-size@\var{max_size}},
\code{min_size}\index{min-size@\var{min_size}},
\code{aspect}\index{aspect@\var{aspect}},
\code{resizeinc}\index{aspect@\var{resizeinc}},
and
\code{ignore_max_size}\index{ignore-max-size@\var{ignore_max_size}},
\code{ignore_min_size}\index{ignore-min-size@\var{ignore_min_size}},
\code{ignore_aspect}\index{ignore-aspect@\var{ignore_aspect}},
\code{ignore_resizeinc}\index{ignore-aspect@\var{ignore_resizeinc}},
may be used to override application-supplied size hints. The four
first ones are tables with the fields \var{w} and \var{h}, indicating
the width and height size hints in pixels, and the latter ignore
winprop is a boolean. 

Finally, the boolean
\code{userpos}\index{userpos@\var{userpos}} option may be used to
override the \code{USPosition} flag of the size hints. Normally,
when this flag is set, Ion tries to respect the supplied window
position more than when it is not set. Obviously, this makes sense
only for floating windows.


\subsection{Classes, roles and instances}
\label{sec:classesrolesinstances}

The identification information supported are
\var{class}\index{class@\var{class}!winprop},
\var{role}\index{role@\var{role}!winprop},
\var{instance}\index{instance@\var{instance}!winprop},
\var{name}\index{name@\var{name}!winprop},
\var{is_transient}\index{is-transient@\var{is_transient}!winprop}, and
\var{is_dockapp}\index{is-dockapp@\var{is_dockapp}!winprop}.
It is not necessary to specify all of these fields.
The first three are strings, and must exactly match the
corresponding information obtained from the window's properties.
The \var{name} field is a Lua-style regular expression matched against
the window's title. The \var{is_transient} field is a boolean that can
be used to include or exclude transients only, while the \var{is_dockapp}
field is set by Ion for the dock windows of Window Maker dockapp protocol
dockapps. Usually this is the only information available for these 
\emph{icon} windows. 

Ion looks for a matching winprop in the order listed by the following
table. An 'E' indicates that the field must be set in the winprop
and it must match the window's corresponding property exactly or, in
case of \var{name}, the regular expression must match the window
title. An asterisk '*' indicates that a winprop where the field is
not specified (or is itself an asterisk in case of the first three
fields) is tried.

\begin{center}
\begin{tabular}{llll}
  \tabhead{\var{class} & \var{role} & \var{instance} & other}
  E	       & E          & E              & E \\
  E	       & E          & E              & * \\
  E	       & E          & *              & E \\
  E	       & E          & *              & * \\
  E	       & *          & E              & E \\
  E	       & *          & E              & * \\
  E	       & *          & *              & E \\
  \vdots       & \vdots     & \vdots         & etc. \\
\end{tabular}
\end{center}

If there are multiple matching winprops with the same
\var{class}, \var{role} and \var{instance}, but other information
different, the most recently defined one is used.


\subsection{Finding window identification}

The 'Window info' context menu entry (\key{Mod1+M} or \key{Button3} on a tab)
can be used to list the identification information required to set winprops
for a window and all the transient windows managed within it. 

\index{xprop} 
Another way to get the identification information is to use \command{xprop}.
Simply run To get class and instance, simply run \command{xprop WM_CLASS}
and click on the particular window of interest. The class is the latter of
the strings while the instance is the former.  To get the role -- few
windows have this property -- use the command \command{xprop WM_ROLE}. 
This method, however, will not work on transients. 

\index{transient}
So-called ``transient windows'' are usually short-lived dialogs (although
some programs abuse this property) that have a parent window that they are
``transient for''. On tiled workspaces Ion displays these windows 
simultaneously with the parent window at the bottom of the same frame.
Unfortunately \command{xprop} is stupid and can't cope with this situation,
returning the parent window's properties when the transient is clicked on.
For this reason you'll have to do a little extra work to get the properties
for that window.\footnote{There's a patch to \command{xprop} to
fix this, but nothing seems to be happening with respect to including it in 
XFree86.}

Finally, it should be mentioned that too many authors these days
``forget'' to set this vital identification to anything meaningful:
everything except name is the same for all of the program's 
windows, for example. Some other programs only set this information
after the window has been mapped, i.e. the window manager has been
told to start managing it, which is obviously too late. 
Gtk applications in particular are often guilty on both counts.


\subsection{Some common examples}

\subsubsection{Acrobat Reader}

The following is absolutely necessary for Acrobat reader:

\begin{verbatim}
defwinprop{
    class = "AcroRead",
    instance = "documentShell",
    acrobatic = true,
}
\end{verbatim}

\subsubsection{Forcing newly created windows in named frames}

The following winprop should place xterm started with command-line parameter
\mbox{\code{-name sysmon}} and running a system monitoring program in a
particular frame:
\begin{verbatim}
defwinprop{
    class = "XTerm",
    instance = "sysmon",
    target = "sysmonframe",
}
\end{verbatim}

For this example to work, we have to somehow create a frame named
\codestr{sysmonframe}. One way to do this is to make the following
call in the \key{Mod1+F3} Lua code query:

\begin{verbatim}
mod_query.query_renameframe(_)
\end{verbatim}

Recall that \code{_} points to the multiplexer (frame or screen) in which 
the query was opened. Running this code should open a new query prefilled
with the current name of the frame. In our example we would change the 
name to \codestr{sysmonframe}, but we could just as well have used the 
default name formed from the frame's class name and an instance number.


\section{The statusbar}
\label{sec:statusbar}

The \file{mod\_statusbar} module provides a statusbar that adapts to 
layouts of tilings, using only the minimal space needed. Ion only 
supports one adaptive ``status display'' object per screen, so this
statusbar is mutually exclusive with the embedded mode of \file{mod\_dock} 
docks. 

The statusbar is configured in \file{cfg\_statusbar.lua}. Typically,
the configuration consists of two steps: creating a statusbar with
\fnref{mod\_statusbar.create}, and then launching the separate
\command{ion-statusd} status daemon process with 
\fnref{mod_statusbar.launch_statusd}. This latter phase is done
automatically, if it was not done by the configuration file, but
the configuration file may pass extra parameters to \command{ion-statusd}
monitors. (See Section \ref{sec:statusd} for more information on
writing \command{ion-statusd} monitors.)

A typical \file{cfg\_statusbar.lua} configuration might look as follows:


\begin{verbatim}
-- Create a statusbar
mod_statusbar.create{
    screen = 0,     -- First screen, 
    pos = 'bl',     -- bottom left corner
    systray = true, -- Swallow systray windows

    -- The template
    template = "[ %date || load:% %>load || mail:% %>mail_new/%>mail_total ]"
               .. " %filler%systray",
}

-- Launch ion-statusd. 
mod_statusbar.launch_statusd{
    -- Date meter
    date={
        -- ISO-8601 date format with additional abbreviated day name
        date_format='%a %Y-%m-%d %H:%M',
    },      
}
\end{verbatim}


\subsection{The template}

The template specifies what is shown on the statusbar; for information
on the other options to \fnref{mod_statusbar.create}, see the reference. 
Strings of the form \codestr{\%spec} tokens specially interpreter by
the statusbar; the rest appears verbatim. The \code{spec} typically
consists of the name of the value/meter to display (beginning with a latin
alphabet), but may be preceded by an alignment specifier and a number
specifying the minimum width. The alignment specifiers are: \codestr{>}
for right, \codestr{<} for left,  and \codestr{|} for centring. Additionally,
space following \codestr{\%} (that is, the string \codestr{\% }), adds
``stretchable space'' at that point. The special string \codestr{\%filler}
may be used to flush the rest of the template to the right end of 
the statusbar. 

The stretchable space works as follows: \file{mod\_statusbar} remembers
the widest string (in terms of graphical presentation) that it has
seen for each meter, unless the width has been otherwise constrained.
If there is stretchable space in the template, it tries to make the
meter always take this much space, by stretching any space found in
the direction indicated by the alignment specifier: the opposite
direction for left or right alignment, and both for centring.

\subsection{The systray}

The special \codestr{\%systray} and \codestr{\%systray\_*}
(\codestr{*} varying) monitors indicate where to place system tray 
windows.  There may be multiple of these. KDE-protocol system tray
icons are placed in \codestr{\%systray} automatically, unless disabled 
with the \var{systray} option. Otherwise the \var{statusbar} winprop may
be used to place any window in any particular \codestr{\%systray\_*}.

\subsection{Monitors}

The part before the first
underscore of each monitor name, describes the script/plugin/module
that provides the meter, and any configuration should be passed
in the a corresponding sub-table \fnref{mod_statusbar.launch_statusd}.
Ion comes with date, load and mail (for plain old mbox) 
\command{ion-statusd} monitor scripts. More may be obtained from 
the scripts repository \cite{scripts}. These included scripts 
provide the following monitors and their options

\subsubsection{Date}

Options: \var{date_format}: The date format in as seen above, 
in the usual \code{strftime} format. \code{formats}: table of
formats for additional date monitors, the key being the name
of the monitor (without the \codestr{date\_} prefix).

Monitors: \codestr{date} and other user-specified ones with the
\codestr{date\_} prefix.


\subsubsection{Load}

Options: \var{update_interval}: Update interval in milliseconds
(default 10s). \var{important_threshold}: Threshold above which 
the load is marked as important (default 1.5), so that the 
drawing engine may be suitably hinted. \var{critical_threshold}: 
Threshold above which  the load is marked as critical (default 4.0).


Monitors: \codestr{load} (for all three values), 
\codestr{load\_1min}, \codestr{load\_5min} and \codestr{load\_15min}.


\subsubsection{Mail}

Options: \var{update_interval}: Update interval in milliseconds
(default 1min). \var{mbox}: mbox-format mailbox location
(default \verb!$MAIL!). 
\var{files}: list of additional mailboxes, the key giving the 
name of the monitor.

Monitors: \codestr{mail\_new}, \codestr{mail\_unread},
\codestr{mail\_total}, and corresponding
\codestr{mail\_*\_new}, \codestr{mail\_*\_unread}, and \codestr{mail\_*\_total}
for the additional mailboxes (\codestr{*} varying).




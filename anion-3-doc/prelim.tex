
\chapter{Preliminaries: Key concepts and relations}
\label{chap:prelim}

The purpose of this chapter to explain some of key concepts and
relations you need to understand before reading the following
chapters. These include modules explained in section \ref{sec:modules}
and the Ion class and object hierarchies, section \ref{sec:objects}.


\section{Modules}
\label{sec:modules}

Ion has been designed so that the 'ion' executable only implements some
basic services on top of which very different kinds of window managers
could be build by loading the appropriate 'modules'. On modern system
these modules are simply dynamically loaded \file{.so} libraries. On 
more primitive systems, or if you want to squeeze total size of the 
executable and libraries, the modules can optionally be statically 
linked to the main binary, but must nevertheless be loaded with the
\fnref{dopath} function. Modules may also include Lua code.

Currently Ion provides the following modules:

\begin{description}
    \item[\file{mod\_tiling}] Tilings for workspaces of the original tiled
       Ion kind.
    \item[\file{mod\_query}] Queries (for starting programs and so on)
      and message boxes.
    \item[\file{mod\_menu}] Support for menus, both pull-down and
      keyboard-operated in-frame menus.
    \item[\file{mod\_statusbar}] Module that implements a statusbar that
      can be adaptively embedded in each workspace's layout.
    \item[\file{mod\_dock}] Module for docking Window Maker dock-apps.
      The dock can both float and be embedded as the statusbar.
    \item[\file{mod\_sp}] This module implements a scratchpad frame that can
      be toggled on/off everywhere. Think of the 'console' in some 
      first-person shooters.
    \item[\file{mod\_sm}] Session management support module.
      \emph{Loaded automatically when needed!}
\end{description}

So-called drawing engines are also implemented as a modules,
but they are not discussed here; see chapter \ref{chap:gr}.

The stock configuration for the \file{ion3} executable loads all of the 
modules mentioned above except \file{mod\_dock}.
The stock configuration for the \file{pwm3} executable (which differs
from the \file{ion3} executable in a few configuration details)
loads another set of modules.



\section{Class and object hierarchies}
\label{sec:objects}

While Ion does not not have a truly object-oriented design
\footnote{the author doesn't like such artificial designs},
things that appear on the computer screen are, however, quite
naturally expressed as such ``objects''. Therefore Ion implements
a rather primitive OO system for these screen objects and some
other things. 

It is essential for the module writer to learn this object
system, but also people who write their own binding configuration files
necessarily come into contact with the class and object hierarchies
-- you need to know which binding setup routines apply where, 
and what functions can be used as handlers in which bindings.
It is the purpose of this section to attempt to explain these 
hierarchies. If you do not wish the read the full section, at least
read the summary at the end of it, so that you understand the very
basic relations.

For simplicity we consider only the essential-for-basic-configuration
Ioncore, \file{mod\_tiling} and \file{mod\_query} classes. 
See Appendix \ref{app:fullhierarchy} for the full class hierarchy visible
to Lua side.

\subsection{Class hierarchy}

One of the most important principles of object-oriented design methodology
is inheritance; roughly how classes (objects are instances of classes)
extend on others' features. Inheritance gives rise to class hierarchy.
In the case of single-inheritance this hierarchy can be expressed as a
tree where the class at the root is inherited by all others below it
and so on. Figure \ref{fig:classhierarchy} lists out the Ion class 
hierarchy and below we explain what features of Ion the classes 
implement.

\begin{figure}
\begin{htmlonly}
\docode % latex2html kludge
\end{htmlonly}
\begin{verbatim}
    Obj
     |-->WRegion
     |    |-->WClientWin
     |    |-->WWindow
     |    |    |-->WMPlex
     |    |    |    |-->WFrame
     |    |    |    `-->WScreen
     |    |    |         `-->WRootWin
     |    |    `-->WInput (mod_query)
     |    |         |-->WEdln (mod_query)
     |    |         `-->WMessage (mod_query)
     |    |-->WGroup
     |    |    |-->WGroupWS
     |    |    `-->WGroupCW
     |    `-->WTiling (mod_tiling)
     `-->WSplit (mod_tiling)
\end{verbatim}
\caption{Partial Ioncore, \file{mod\_tiling} and \file{mod\_query} 
    class hierarchy.}
\label{fig:classhierarchy}
\end{figure}

The core classes:

\begin{description}
  \item[\type{Obj}]\indextype{Obj}
    Is the base of Ion's object system.

  \item[\type{WRegion}]\indextype{WRegion}
    is the base class for everything corresponding to something on the
    screen. Each object of type \type{WRegion} has a size and  position
    relative to the parent \type{WRegion}. While a big part of Ion 
    operates on these instead of more specialised classes, \type{WRegion}
    is a ``virtual''  base class in that there are no objects of ``pure''
    type \type{WRegion}; all concrete regions are objects of some class 
    that inherits \type{WRegion}.

  \item[\type{WClientWin}]\indextype{WClientWin} is a class for
    client window objects, the objects that window managers are
    supposed to manage.

  \item[\type{WWindow}]\indextype{WWindow} is the base class for all
    internal objects having an X window associated to them
    (\type{WClientWins} also have X windows associated to them).
    
  \item[\type{WMPlex}] is a base class for all regions that ``multiplex'' 
    other regions. This means that of the regions managed by the multiplexer,
    only one can be displayed at a time. 
  
  \item[\type{WScreen}]\indextype{WScreen} is an instance of \type{WMPlex}
    for screens.
    
  \item[\type{WRootWin}]\indextype{WRootWin} is the class for
    root windows\index{root window} of X screens\index{screen!X}.
    It is an instance of \type{WScreen}.
    Note that an ``X screen'' or root window is not necessarily a
    single physical screen\index{screen!physical} as a root window
    may be split over multiple screens when ugly hacks such as 
    Xinerama\index{Xinerama} are used. (Actually there can be only 
    one root window when Xinerama is used.) 
    
  \item[\type{WFrame}]\indextype{WFrame} is the class for frames.
    While most Ion's objects have no graphical presentation, frames 
    basically add to \type{WMPlex}es the decorations around client 
    windows (borders, tabs).
    
  \item[\type{WGroup}]\indextype{WGroup} is the base class for groups.
    Particular types of groups are workspaces 
    (\type{WGroupWS}\indextype{WGroupWS})
    and groups of client windows
    (\type{WGroupCW}\indextype{WGroupCW}).
\end{description}


Classes implemented by the \file{mod\_tiling} module:

\begin{description}
  \item[\type{WTiling}]\indextype{WTiling} is the class for tilings
    of frames.
  \item[\type{WSplit}]\indextype{WSplit} (or, more specifically, classes
    that inherit it) encode the \type{WTiling} tree structure.
\end{description}


Classes implemented by the \file{mod\_query} module:

\begin{description}
  \item[\type{WInput}]\indextype{WInput} is a virtual base class for the
    two classes below.
  \item[\type{WEdln}]\indextype{WEdln} is the class for the ``queries'',
    the text inputs that usually appear at bottoms of frames and sometimes
    screens. Queries are the functional equivalent of ``mini buffers'' in
    many text editors.
  \item[\type{WMessage}]\indextype{WMessage} implements the boxes for 
    warning and other messages that Ion may wish to display to the user. 
    These also usually appear at bottoms of frames.
\end{description}

There are also some other ``proxy'' classes that do not refer
to objects on the screen. The only important one of these for
basic configuration is \type{WMoveresMode} that is used for
binding callbacks in the move and resize mode.


\subsection{Object hierarchies: \type{WRegion} parents and managers}

\subsubsection{Parent--child relations}
Each object of type \type{WRegion} has a parent and possibly a manager
associated to it. The parent\index{parent} for an object is always a 
\type{WWindow} and for \type{WRegion} with an X window (\type{WClientWin},
\type{WWindow}) the parent \type{WWindow} is given by the same relation of
the X windows. For other \type{WRegion}s the relation is not as clear.
There is generally very few restrictions other than the above on the
parent---child relation but the most common is as described in
Figure \ref{fig:parentship}.

\begin{figure}
\begin{htmlonly}
\docode % latex2html kludge
\end{htmlonly}
\begin{verbatim}
    WRootWins
     |-->WGroupWSs
     |-->WTilings
     |-->WClientWins in full screen mode
     `-->WFrames
          |-->WGroupCWs
          |-->WClientWins
          |-->WFrames for transients
          `-->a possible WEdln or WMessage
\end{verbatim}
\caption{Most common parent--child relations}
\label{fig:parentship}
\end{figure}

\type{WRegion}s have very little control over their children as a parent.
The manager\index{manager} \type{WRegion} has much more control over its
managed \type{WRegion}s. Managers, for example, handle resize requests,
focusing and displaying of the managed regions. Indeed the manager---managed
relationship gives a better picture of the logical ordering of objects on
the screen. Again, there are generally few limits, but the most common
hierarchy is given in Figure \ref{fig:managership}. Note that sometimes
the parent and manager are the same object and not all regions may have
a manager, but all non-screen regions have a parent---a screen if not 
anything else.

\subsubsection{Manager--managed relations}

\begin{figure}
\begin{htmlonly}
\docode % latex2html kludge
\end{htmlonly}
\begin{verbatim}
    WRootWins
     |-->WGroupCWs for full screen WClientWins
     |    |-->WClientWins
     |    `-->WFrames for transients (dialogs)
     |         `--> WClientWin
     |-->WGroupWSs for workspaces
     |    |-->WTiling
     |    |    |-->WFrames
     |    |    |    `-->WGroupCWs (with contents as above)
     |    |    `-->possibly a WStatusBar or WDock
     |    |-->WFrames for floating content
     |    |-->possibly a WEdln, WMessage or WMenu
     |    `-->possibly a WStatusBar or WDock (if no tiling)
     `-->WFrames for sticky stuff, such as the scratchpad
\end{verbatim}
\caption{Most common manager--managed relations}
\label{fig:managership}
\end{figure}

Note that a workspace can manage another workspace. This can be
achieved with the \fnref{attach_new} function, and allows you to nest
workspaces as deep as you want.

%Note how the \type{WClientWin}s managed by \type{WFloatFrame}s don't have
%transients managed by them. This is because WFloatWSs choose to handle
%transients differently (transients are put in separate frames like normal
%windows).

\subsection{Summary}

In the standard setup, keeping queries, messages and menus out of
consideration:

\begin{itemize}
  \item The top-level objects that matter are screens and they correspond
    to physical screens. The class for screens is \type{WScreen}.
  \item Screens contain (multiplex) groups (\type{WGroup}) and other 
    objects, such as \type{WFrames}. Some of these are mutually exclusive
    to be viewed at a time.
  \item Groups of the specific kind \type{WGroupWS} often contain a
    \type{WTiling} tiling for tiling frames (\type{WFrame}), but 
    groups may also directly contain floating frames.
  \item Frames are the objects with decorations such as tabs and borders.
    Frames contain (multiplex) among others (groups of) client windows, 
    to each of which corresponds a tab in the frame's decoration. Only 
    one client window (or other object) can be shown at a time in each 
    frame. The class for client windows is \type{WClientWin}.
\end{itemize}


